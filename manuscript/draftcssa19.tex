\documentclass{article}
\usepackage{amsmath}
\usepackage{amssymb}
\usepackage{amsfonts}
%\usepackage{graphics}
%\usepackage{curves}
%\usepackage{tikz}
%\usetikzlibrary{backgrounds}
%\usetikzlibrary{snakes} \documentclass{report}
%\usepackage{amsmath}
%\usepackage{amssymb}
%\usepackage{amsfonts}
%\usepackage{graphics}
%\usepackage{curves}
%\usepackage{tikz}
%\usetikzlibrary{backgrounds}
%\usetikzlibrary{snakes}
%\usepackage{setspace}
%\doublespacing
%\usepackage{lscape}
%\usepackage{booktabs}
%\usepackage{longtable}
\usepackage{hyperref}
\usepackage{url}
\usepackage{color}

\author{ABMSPECSIG Group}
\title{Wealth dynamics in the presence of network structure and primitive cooperation}
\bibliographystyle{plain} 
\begin{document}

\maketitle
\begin{abstract}
We study wealth accumulation dynamics in a population of heterogeneously mixed agents with a capacity for a certain primitive form of cooperation enabled by static network structures. Despite their simplicity, the stochastic dynamics generate inequalities in wealth reminiscent of real world social systems even in a fully mixed population. A simple form of cooperation is introduced and is shown to enhance the viability of agents By embedding such dynamics in a network, the impact of social structures on the origins and persistence of inequality can be teased out easily. The models developed here complement traditional modeling approaches based on grid worlds.   

\end{abstract}
\section{Introduction}
In recent years, concerns and debates surrounding wealth inequality and socioeconomic mobility have been one of the few unifying issues dominating the extremely polarized public spheres of the Global North. While economic and political inquality used to be discussed in not so mainstream heterodox economics circles, the contentious discussions on the topic since the publication of Piketty's book suggests a lack of consensus about very basic foundational questions like the origins and persistence of economic inequality. Not surprisingly, traditional theories and tools of macro and micro economics are now being diagnosed for its limitations. Simultaneously, insights from related disciplines along with novel scientific inquiry models not usually associated with traditional econometrics are being taken more seriously. The current work shares this spirit by integrating substantive ideas from anthropology, economic sociology and urban sociology and using modelling approaches from computational social science and analytical sociology to understand the origins of inequality in a simple model of wealth dynamics in the presence of social structures.    

The current work originated in our attempts to incorporate and tease out the effects of social structures in simple models of wealth dynamics in the presence of environmental stochasticity, and a simple form of resource pooling. These models were inspired by our search for analogs of grid world agent based models of Srikanth and Milton \footnote{Include proper citation of SriMil here} where foraging agents that pool their resources were shown to on-average outperform non-pooling agents. While interaction or mixing of agent population is achieved through foraging in the original model, the model presented here achieves this by partially mixing the agents via a static network structure. 

The original model drew its inspiration from historical sociology, in Katz's influential study of middle of 19th century Hamilton, Canada. Retaining this original motivation, we draw additional inspiration from economic sociology, in the work of Grannovetter; urban sociology, in the work of Sampson and others; and in anthropology, in the work on cooperation in small to medium scale societies.  
 
%The current plan is start with economics economic sociology and wealth dynamic models. The former is discussed in Jackson's review, the latter in rvenkat's CSSA18 paper. Then we move on to how wealth dynamics and inequality questions matter also to other fields like anthropology and sociology. Wealh dynamics in such situations can also be addressed using economic models and is the focus of our manuscript.


%The general idea is to discuss how/why social networks matter wealth accumulation dynamics. We have references from anthropology. We have references in economic sociology(jackson).Sampson's work and Venkatesh's work can be used when talking about urban poverty and urban sociology. Discuss how in economics, wealth accumulation dynamics are mostly discussed at the macroeconomic level. In this sense, this work (and the work presented last year) are novel in this regard.  

%% Sampson's work seem to be crucial for making the claim that social structures and neighborhood effects are important determinants of economic outcomes. 

% Wealth accumulation models all start from fancy macroeconomic models and may not be well suited for studying wealth accumulation processes under poverty where high level of stochasticity seem to matter. 

%While we are interested in more sophisticated models, we are more interested in extracting the effects of social network structure in simple models of cooperative wealth accumulation. The proto-institution is definetely borrowed from SriMil. In order to study more elaborate and economically rigorous wealth dynamics, we need to be able sort of the effects of network structure in simple stylized network models. The work on ER models is expected to be first of our analysis. Eventually, we plan to get to more realistic wealth accumulation models of interest to economists but for now this is all we have.  

\cite{power2018cooperation,power2018}

\cite{koster2019,koster2014,koster2015,bogerhoff2015}

\cite{smith2019,nolin2012}



~\cite{ch1as_hdbk,hedstrom2011oxford,martin_lee}

~\cite{ch2as_hdbk,ch11as_hdbk,granovetter2005,hedstrom2018}

%In Katz's book, the differing life histories of individuals in Hamilton, Canada suggest that luck played a role in deterministic life outcomes. Katz argues that social mobility makes sense only in the context of social structure

%Katz's chapter 2. His main argument there is that the structure of inequality rpresent in Hamilton, Canada is representative of inequality present in all other north american cities of that era. The persistence of inequality can be see in the persistence of four kinds of social structures. 1) occupational structure, 2) division of wealth and proportion of the wealthy, 3) social and demographic identity of different economic ranks, and 4) distance between people of different socio-economic ranks or social stratification. He talks about peristence of networks of wealth and power. Points out strong interconnection between property, political power and social status. Argues that economic stratification and its crystallization leads to crystallization of social and other subjective stratifications.  

\section{Model}
~\cite{redner2001guide}
%Use chapter 3 of redner's book here. 
%Also, the network model presented here seem to be one of a kind. Typicall dynamcics on networks are deterministic in nature. The models discussed here is different from deterministic dynamic processes and epidemic processes, and diffusion type processes (percolation processes). The models are stochastic dynamic systems on networks. Such social processes are studied using Stochastic Actor Oriented Models (SOAM) Snidjers. ALso use Kolacyzk and Csardi's discussion as reference. Social influence models..... and others) This is not directly about influence but more about cooperation. Need to find a nice way of talking about this. 

%In the model section, we motivate our current specnet model as attempts to capture specific aspects of historical sociology and urban sociology of Katz and Samson respectively.  

\section{Analysis}

\section{Conclusion and Future Work}
%what kinds of real world systems are we talking about?This section should write itself. Just a brief discussion of phase 2+ and integration with spescape should suffice 
%I am actually confused about whether such a purely preliminary work can say antyhing about the real world. Do we really want to say anything at all about the real world. What can ER graphs say about the real world?  

%We conclude by discussing features of social processes and mechanisms argued for by both Sampson and Katz that are currently not present in the model but important and part of the vision for where our model is headed. 

%We also include the goal of seamlessly going from a model with no geography to a model with geography. 

\bibliography{ref_cssa19}
\end{document}
