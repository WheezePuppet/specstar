\documentclass{article}
\usepackage{amsmath}
\usepackage{amssymb}
\usepackage{amsfonts}
%\usepackage{graphics}
%\usepackage{curves}
%\usepackage{tikz}
%\usetikzlibrary{backgrounds}
%\usetikzlibrary{snakes} \documentclass{report}
%\usepackage{amsmath}
%\usepackage{amssymb}
%\usepackage{amsfonts}
%\usepackage{graphics}
%\usepackage{curves}
%\usepackage{tikz}
%\usetikzlibrary{backgrounds}
%\usetikzlibrary{snakes}
%\usepackage{setspace}
%\doublespacing
%\usepackage{lscape}
%\usepackage{booktabs}
%\usepackage{longtable}
\usepackage{hyperref}
\usepackage{url}
\usepackage{color}

\author{ABMSPECSIG Group}
\title{Wealth dynamics in the presence of network structure and primitive cooperation}
\bibliographystyle{plain} 
\begin{document}

\maketitle
\begin{abstract}
We study wealth accumulation dynamics in a population of heterogeneously mixed agents with a capacity for a certain primitive form of cooperation enabled by static network structures. Despite their simplicity, the stochastic dynamics generate inequalities in wealth reminiscent of real world social systems even in a fully mixed population. A simple form of cooperation is introduced and is shown to enhance the viability of agents By embedding such dynamics in a network, the impact of social structures on the origins and persistence of inequality can be teased out easily. The models developed here complement traditional modeling approaches based on grid worlds.   

\end{abstract}
\section{Introduction}

~\cite{ch1as_hdbk,hedstrom2011oxford,martin_lee}

~\cite{ch2as_hdbk,ch11as_hdbk,granovetter2005,hedstrom2018}

\section{Model}
~\cite{redner2001guide}

\section{Analysis}

\section{Conclusion and Future Work}
%what kinds of real world systems are we talking about?
\bibliography{ref_cssa19}
\end{document}
