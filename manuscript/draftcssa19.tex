\documentclass{article}
\usepackage{amsmath}
\usepackage{amssymb}
\usepackage{amsfonts}
%\usepackage{graphics}
%\usepackage{curves}
%\usepackage{tikz}
%\usetikzlibrary{backgrounds}
%\usetikzlibrary{snakes} \documentclass{report}
%\usepackage{amsmath}
%\usepackage{amssymb}
%\usepackage{amsfonts}
%\usepackage{graphics}
%\usepackage{curves}
%\usepackage{tikz}
%\usetikzlibrary{backgrounds}
%\usetikzlibrary{snakes}
%\usepackage{setspace}
%\doublespacing
%\usepackage{lscape}
%\usepackage{booktabs}
%\usepackage{longtable}
\usepackage{hyperref}
\usepackage{url}
\usepackage{color}

\author{ABMSPECSIG Group}
\title{Wealth dynamics in the presence of network structure and primitive cooperation}
\bibliographystyle{plain} 
\begin{document}

\maketitle
\begin{abstract}
We study wealth accumulation dynamics in a population of heterogeneously mixed agents with a capacity for a certain primitive form of cooperation enabled by static network structures. Despite their simplicity, the stochastic dynamics generate inequalities in wealth reminiscent of real world social systems even in a fully mixed population. A simple form of cooperation is introduced and is shown to enhance the viability of agents By embedding such dynamics in a network, the impact of social structures on the origins and persistence of inequality can be teased out easily. The models developed here complement traditional modeling approaches based on grid worlds.   

\end{abstract}
\section{Introduction}
In recent years, concerns and debates surrounding wealth inequality and socioeconomic mobility have been one of the few unifying issues dominating the extremely polarized public spheres of the Global North. While economic and political inequality used to be discussed in not so mainstream heterodox economics circles, the contentious discussions on the topic within mainstream economics since the publication of Piketty's book~\cite{piketty2017capital} suggests a lack of consensus about very basic foundational questions like the origins and persistence of economic inequality. Not surprisingly, traditional theories and tools of macro and micro economics are now being diagnosed for their limitations. Simultaneously, insights from related disciplines, along with novel scientific inquiry models not usually associated with traditional econometrics, are being taken more seriously. The work presented here shares this spirit by integrating substantive ideas from anthropology, economic sociology~\cite{granovetter2017society} and urban sociology~\cite{sampson2012great} and using modeling approaches from computational social science~\cite{hedstrom2018} and analytical sociology~\cite{hedstrom2011oxford} to understand the origins of inequality in a simple model of wealth dynamics in the presence of social structures.    
% SD: citation for "anthropology," the first element of the last sentence's list? Grief and/or White?

The current work originated in our attempts to incorporate and tease out the effects of social structures in simple models of wealth dynamics in the presence of environmental stochasticity, and a simple form of resource pooling. These models were inspired by our search for analogs of the grid world agent-based models (ABMs) of Friesen and Mudigonda \footnote{Include proper citation of SriMil here} where foraging agents that pool their resources were shown to on-average outperform non-pooling agents. The original model used foraging to mix the population and create opportunities for interaction, and, when certain conditions were met, allowed resource sharing. Our model achieves agent interaction by postulating a static network structure which partially mixes the agents. 

The original model drew its inspiration from historical sociology, in Katz's influential study of middle of 19th century Hamilton, Canada~\cite{katz2013people}. Retaining this original motivation, we draw additional inspiration from economic sociology, in the work of Granovetter~\cite{granovetter2005}; urban sociology, in the work of Sampson~\cite{sampson2002} and others; and in anthropology, in the work on cooperation in small to medium scale societies~\cite{avner1994,white2011kinship}. The geographic and economic scale of the systems, the nature of social actors and time period of interest are all very different from the ones used to develop macroeconomic representative agent models~\cite{benhabib2018}, making the similarity between macroeconomic wealth dynamics models and our models not comparable without further justification. Elaborating on the interplay of conceptual and methodological ideas among these disciplines is beyond the scope of this article. Instead, we anchor this work in economic sociology and revisit the insights from the above disciplines in the article's concluding section.

The important role played by social structures or non-economic structures in determining economic outcomes of individuals in a society is not in doubt~\cite{granovetter2005,jackson_rev2017}. Still, in the absence of a unifying foundation for sociology and economics, the full impact of this two-way interpenetration of economic and social structures is demonstrated only on a case-by-case basis. The language of social and economic networks affords a first principle integration by simplifying the non-trivial concept of social and economic structure~\cite{martin_lee} to only dyadic relations. 

As mentioned earlier, macroeconomic models are ill-suited to the study of collective phenomena at an intermediate level. Hence, alternative explanations of aggregate phenomena that match the expressiveness of economic models are required. Analytical sociology~\cite{ch1as_hdbk}, with its emphasis on explanation of collective emergent phenomena using mathematically formulated social mechanisms~\cite{ch2as_hdbk,ch11as_hdbk} anchored at the individual level, is an ideal candidate for this purpose. 

Models used in scientific inquiry serve a specific purpose. In this work, the goal is to construct simple toy models that reproduce non-trivial wealth inequality distributions in the presence and absence of primitive forms of cooperation, clearly delineating the role of network structure in generating wealth inequality. We make no suggestions that these models \textit{explain} the phenomena of interest; we are only interested in constructing the simplest possible models, with no detailed empirical grounding, but with the potential to generate realistic looking inequality distributions. In doing so, we aspire to shed light on the \textit{true} social and economic mechanisms underlying the genesis and persistence of economic inequality. 

%Briefly explain the model here and then move on to the next section. . 


 
%The current plan is start with economics economic sociology and wealth dynamic models. The former is discussed in Jackson's review, the latter in rvenkat's CSSA18 paper. Then we move on to how wealth dynamics and inequality questions matter also to other fields like anthropology and sociology. Wealh dynamics in such situations can also be addressed using economic models and is the focus of our manuscript.


%The general idea is to discuss how/why social networks matter wealth accumulation dynamics. We have references from anthropology. We have references in economic sociology(jackson).Sampson's work and Venkatesh's work can be used when talking about urban poverty and urban sociology. Discuss how in economics, wealth accumulation dynamics are mostly discussed at the macroeconomic level. In this sense, this work (and the work presented last year) are novel in this regard.  

%% Sampson's work seem to be crucial for making the claim that social structures and neighborhood effects are important determinants of economic outcomes. 

% Wealth accumulation models all start from fancy macroeconomic models and may not be well suited for studying wealth accumulation processes under poverty where high level of stochasticity seem to matter. 

%While we are interested in more sophisticated models, we are more interested in extracting the effects of social network structure in simple models of cooperative wealth accumulation. The proto-institution is definitely borrowed from SriMil. In order to study more elaborate and economically rigorous wealth dynamics, we need to be able sort of the effects of network structure in simple stylized network models. The work on ER models is expected to be first of our analysis. Eventually, we plan to get to more realistic wealth accumulation models of interest to economists but for now this is all we have.  





%In Katz's book, the differing life histories of individuals in Hamilton, Canada suggest that luck played a role in deterministic life outcomes. Katz argues that social mobility makes sense only in the context of social structure

%Katz's chapter 2. His main argument there is that the structure of inequality represent in Hamilton, Canada is representative of inequality present in all other north American cities of that era. The persistence of inequality can be see in the persistence of four kinds of social structures. 1) occupational structure, 2) division of wealth and proportion of the wealthy, 3) social and demographic identity of different economic ranks, and 4) distance between people of different socio-economic ranks or social stratification. He talks about persistence of networks of wealth and power. Points out strong interconnection between property, political power and social status. Argues that economic stratification and its crystallization leads to crystallization of social and other subjective stratifications.  

\section{Model}
~\cite{redner2001guide}
%Use chapter 3 of redner's book here. 
%Also, the network model presented here seem to be one of a kind. Typically dynam ics on networks are deterministic in nature. The models discussed here is different from deterministic dynamic processes and epidemic processes, and diffusion type processes (percolation processes). The models are stochastic dynamic systems on networks. Such social processes are studied using Stochastic Actor Oriented Models (SOAM) Snidjers. ALso use Kolacyzk and Csardi's discussion as reference. Social influence models..... and others) This is not directly about influence but more about cooperation. Need to find a nice way of talking about this. 

%In the model section, we motivate our current specnet model as attempts to capture specific aspects of historical sociology and urban sociology of Katz and Samson respectively.  

%Stephen thinks that a subsection of the Model section should be about the
%simulation code. It won't be long; just enough to explain how the mathematical
%model just described is represented and executed in Julia, including some
%answers to non-obvious questions that arise when considering implementation.

\section{Analysis}

\section{Conclusion and Future Work}
%what kinds of real world systems are we talking about?This section should write itself. Just a brief discussion of phase 2+ and integration with spescape should suffice 
%I am actually confused about whether such a purely preliminary work can say antyhing about the real world. Do we really want to say anything at all about the real world. What can ER graphs say about the real world?  

%We conclude by discussing features of social processes and mechanisms argued for by both Sampson and Katz that are currently not present in the model but important and part of the vision for where our model is headed. 

%We also include the goal of seamlessly going from a model with no geography to a model with geography. 

% I am moving the discussion of connections to anthropology to this section 

\cite{power2018cooperation,power2018}

\cite{koster2019,koster2014,koster2015,bogerhoff2015}

\cite{smith2019,nolin2012}


\bibliography{ref_cssa19}
\end{document}
