Figure~\ref{fig:giniHistory}

\subsection{Gini coefficient}

    The Gini coefficient  of the agent population as seen in the top plot is influenced by two distinct dynamics: loss/accumulation of wealth and the formation/death of proto-institutions. Over the course of Stage 1 and the beginning of Stage 3, this change in agent wealth is exclusively responsible for changes in the Gini coefficient . As agents accumulate wealth over Stage 1 and 2, the size of wealth differentials shrinks relative to absolute agent wealth, leading to the declining Gini coefficient . The opposite effect occurs during Stage 3 as agent starvation leads to the relative growth of these wealth differentials. The variability in starvation rates further contributes to the increasing Gini coefficient .

    In addition, the proto-institution formation and proto-institution death respectively, indicated by the bottom plot, influence the Gini coefficient  during Stage 2 and the end of Stage 3. As expected, the formation of proto-institutions during Stage 2 contributes to declining Gini coefficient  as the constituent agents of each proto-institution have equivalent wealth values and represent coalitions of perfect economic equality. Accordingly, the death of proto-institutions, beginning around Iteration 20, contributes to increase the Gini coefficient by removing the proto-institution’s diminishing effect on the system’s inequality. 

Note: As the population size decreases over the starvation period the Gini coefficient  becomes increasingly unstable and susceptible to fluctuations in agent wealth.
\subsection{Life expectancy}

% Plots from 7/12 showing you best get in a proto if lambda and white noise 
%   are high


% Wealth histogram is steady ?

